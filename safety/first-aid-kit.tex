%!TEX encoding = UTF-8 Unicode
\documentclass[a4paper,
	twoside,
	russian]{article}
\usepackage[T2A]{fontenc}
\usepackage[utf8x]{inputenc}
\usepackage[a4paper,
	left=3cm,
	right=0.5cm,
	top=0.5cm,
	bottom=0.5cm]{geometry}
\usepackage[russian]{babel}
\usepackage{enumitem}

\begin{document}
\section{Состав аптечки первой помощи}

\begin{tabular}{| l | l |}
	\hline
	\textbf{Препарат} & \textbf{Количество} \\ \hline
	Plum Eye Wash pH Neutral & 1 \\ \hline
	Plum Eye Wash 4604 & 1 \\ \hline
	Напалечники & 15 \\ \hline
	Нашатырь & 1 \\ \hline
	Гемостатическая губка & 3 \\ \hline
	Биоклей БФ-6 & 3 \\ \hline
	Вата стерильная & 1 \\ \hline
	Турникет & 1 \\ \hline
	Анальгин & 2 \\ \hline
	Аспирин & 2 \\ \hline
	Уголь активированный & 1 \\ \hline
	Пантенол & 1 \\ \hline
	Пластырь & 1 \\ \hline
	Бинт стерильный & 1 \\ \hline
	Борный спирт & 1 \\ \hline
\end{tabular}


\section{Справочные данные по препаратам}

\begin{itemize}[noitemsep]
\item \textbf{Plum Eye Wash pH Neutral} - средство для нейтрализации
кислот и щелочей при попадании в глаза.
(https://shop.vostok.ru/catalog/drugoe/dovrachebnaya-pomosch/promyvanie-glaz/rastvor-plum-rn-neytral-200-ml/)
(Код: 131-0167-01)
\item \textbf{Plum Eye Wash 4604} - нейтральный солевой раствор для
промывания глаз при попадании инородных твёрдых частиц.
(https://shop.vostok.ru/catalog/drugoe/dovrachebnaya-pomosch/promyvanie-glaz/rastvor-plum-ay-voss-4631-500-ml/)
(Код: 141-0041-02)
\item \textbf{Окусалин} - нейтральный солевой раствор для промывания глаз
при попадании инородных частиц. Аналог предыдущего пункта, но продаётся
в аптеках.
\item \textbf{Напалечники} - продаются в аптеках. По нормативам, при порезах,
станочники обязаны на перевязанный палец (или не перевязанный, а с
наклеенным пластырем) одеть напалечник. Делается это по понятной
причине - чтобы руку с висюлькой не намотало на станок.
\item \textbf{Нашатырь} - для приведения пострадавшего в сознание.
\item \textbf{Гемостатическая губка} - для наложения на крупные неглубокие
открытые раны.
\item \textbf{Биоклей БФ-6} - для заклеивания глубоких открытых ран.
\item \textbf{Вата стерильная} - для любых нужд.
\item \textbf{Турникет} - более вменяемая альтернатива жгутам для
перетяжки конечностей.
\item \textbf{Анальгин} - обезболивающее.
\item \textbf{Аспирин} - вместе с предыдущим пунктом даёт мощный
жаропонижающий эффект.
\item \textbf{Уголь активированный} - при отравлениях.
\item \textbf{Пантенол} - при ожогах.
\item \textbf{Пластырь} - при порезах.
\item \textbf{Бинт стерильный} - при ранениях и как замена ваты.
\item \textbf{Борный спирт} - для промывки ран.
\end{itemize}


\end{document}

