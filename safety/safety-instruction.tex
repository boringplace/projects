%!TEX encoding = UTF-8 Unicode
\documentclass[a4paper,
	twoside,
	russian]{article}
\usepackage[T2A]{fontenc}
\usepackage[utf8x]{inputenc}
\usepackage[a4paper,
	left=3cm,
	right=0.5cm,
	top=0.5cm,
	bottom=0.5cm]{geometry}
\usepackage[russian]{babel} % It's needed for Russian language
\usepackage{tabulary} % For wrapping text inside tables

\begin{document}
\title{Техника безопасности при работе за станком}
\date{Июль 2018}
\maketitle

\newpage
\tableofcontents

\clearpage
\section{При работе на станках}


\subsection{Техника безопасности}
\begin{tabulary}{\linewidth}{LL}
\textbf{Требование} & \textbf{Причина} \\
\hline
Работа в помещении допускается только при наличии напарника
&
Наличие напарника необходимо для помощи в случае травм а также для
предупреждения несчастных случаев.
\\ \hline
При осуществлении ППР или уборке станка его необходимо обесточить и
повесить табличку "СТАНОК НЕ РАБОТАЕТ"
&
Неожиданное включение станка может повлечь тяжкие травмы персонала.
\\ \hline
При работе на станке укомплектованном лазером рабочий обязан одевать
защитные очки СЭС-22 (ГОСТ 9411-91).
&
Диффузионное рассеяние лазерного луча тяжко и безвозвратно травмирует
сетчатку глаза.
\\ \hline
К работе на станках допускается только персонал ознакомленный с
соответствующим разделом инструкции по технике безопасности, паспортом
конкретного станка, прошедший обучение работе на конкретном станке и
прошедший проверку знаний инструкции по ТБ не позднее полугода назад.
&
Высокая специфичность работы с разными типами станков и сложность
соблюдения всех требований ТБ.
\\ \hline
Запрещается работать в перчатках.
&
Высокая вероятность наматывания перчаток вместе с их пользователем на
движущиеся части станка.
\\ \hline
Работа на заточных и шлифовальных станках, а также с использованием
заточных и шлифовальных/полировальных приспособлений и средств
должна вестись как с использованием средств защиты глаз, так и с
использованием респиратора.
&
Высокая вероятность заболевания различными лёгочными заболеваниями
типа туберкулёза, силикоза и рака.
\\ \hline
Запрещается складывать инструмент на станок.
&
Возможны повреждения инструмента во время работы станка, а также это
создаёт дополнительные небезопасные условия.
\\ \hline
Запрещается отвлекать рабочего во время работы за станком.
&
Создание травмоопасной ситуации, когда рабочий может отвлечься и
потерять контроль над процессом.
\\ \hline
Форма рабочих, взаимодействующих с оборудованием, имеющим открытые
движущиеся части не должна иметь рукавов, а излишне длинные волосы
должны быть закрыты головным убором.
&
Элементы одежды или волосы могут попасть в подвижные части станка и
привести к травмам.
\\ \hline
Запрещается использовать масла в качестве СОТС/СОЖ.
&
Горячие пары масел вызывают заболевания кожи, почек, печени, а также
отёк лёгких.
\\ \hline
При работе на станке с неизолированной от персонала рабочей зоной с
использованием СОЖ необходимо использовать очки для защиты глаз и
респиратор типа РПГ-67 с фильтрами типа А1.
&
Вдыхание паров многоатомных спиртов, масел и парафинов, входящих в
состав СОЖ, может необратимо травмировать ЦНС.
\\ \hline
\end{tabulary}


\clearpage
\subsection{Обеспечение}

\begin{tabulary}{\linewidth}{LL}
\textbf{Требование} & \textbf{Причина} \\ \hline
Станина/консоль станка должна быть оборудована легкодоступной кнопкой
аварийной остановки.
&
Требуется для обеспечения возможности самостоятельной незамедлительной
остановки оборудования сотрудником, который попал в движущиеся части
станка.
\\ \hline
Рабочее помещение обязано быть укомплектовано аптечкой, качество
комплектации и состояние средств которой подлежит ежемесячной проверке.
&
При работе в неблагоприятных условиях даже малые травмы требуют
тщательной обработки. Также невозможность предусмотреть все
травматические ситуации требует предупредительных мер.
\\ \hline
Рабочая поверхность станков, укомплектованных лазером, должна быть
полностью изолирована от глаз рабочего.
&
Диффузионное рассеяние лазерного луча тяжко и безвозвратно травмирует
сетчатку глаза.
\\ \hline
Инструкция по технике безопасности должна храниться в легко доступном
месте, о котором должны быть оповещены все сотрудники предприятия.
&
Это позволяет избежать непоправимо глупых и опасных поступков.
\\ \hline
\end{tabulary}


\clearpage
\subsection{Поведение в случае непредвиденных ситуаций}
\begin{tabulary}{\linewidth}{LL}
\textbf{Требование} & \textbf{Причина} \\ \hline
Остановить станок.
&
Прекращение дальнейшего травмирования работника.
\\ \hline
В случае тяжких травм (наматывание конечностей на шпиндель, зажим
конечностей между шестернями) вызвать спасателей и скорую.
&
Недостаток знаний в области медицины может усугубить травматизм при
попытке освобождения работника.
\\ \hline
В случае вызова спасателей необходимо подготовить паспорт и чертежи
станка, а также подготовить станок к узловой разборке.
&
Данные ероприятия должны облегчить спасателям задачу освобождения
работника из станка.
\\ \hline
Подготовить к прибытию спасателей необходимые спец. инструменты и
съёмники.
&
Инструментального обеспечения спасателей может не хватить для решения
некоторых задач при разборке станка.
\\ \hline
Предметы, прошившие тем или иным способом полости организма запрещено
пытаться извлекать самостоятельно.
&
Велика вероятность дальнейшего травмирования работника, а также
высоки шансы занести инфекцию.
\\ \hline
\end{tabulary}


\end{document}

